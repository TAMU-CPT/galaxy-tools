\documentclass[]{article}
\usepackage{lmodern}
\usepackage{amssymb,amsmath}
\usepackage{ifxetex,ifluatex}
\usepackage{fixltx2e} % provides \textsubscript
\usepackage[T1]{fontenc}
\usepackage[utf8]{inputenc}

\addtolength{\oddsidemargin}{-.875in}
\addtolength{\evensidemargin}{-.875in}
\addtolength{\textwidth}{1.75in}

\addtolength{\topmargin}{-.875in}
\addtolength{\textheight}{1.75in}

\usepackage{fancyhdr}
\pagestyle{fancy}
\lhead{GenomeA Compliance Report}
\chead{}
\rhead{ {{record_name | texify}} }
\lfoot{}
\cfoot{\thepage}
\rfoot{}



\usepackage{microtype}
\usepackage{hyperref}
\hypersetup{unicode=true,
            pdfborder={0 0 0},
            breaklinks=true}
\urlstyle{same}  % don't use monospace font for urls
\usepackage{longtable,booktabs}
\date{Compiled \today}
\title{GenomeA Compliance Report for {{record_nice_name | texify}}}

\begin{document}
%\pagestyle{plain}
\maketitle
This report details possible issues with your submitted genome annotations.

\section{Required Changes}

The changes detailed in this section are required for acceptance of your
submission.

\subsection{Missing Gene Features}

These coding sequences (``CDS'' in your GenBank file) are missing the
associated gene feature (``gene''). This is required for validation by NCBI's
rules which are encoded in the sequin and tbl2asn programs.


{{ missing_genes_bad }} out of {{ missing_genes_good + missing_genes_bad
}} features are lacking their associated gene feature.

\begin{longtable}{ll}
\hline
Feature ID & Location\\
\hline
\endhead

{{ row.id | texify }} & \texttt{{'{'}}{{row.location}}{{'}'}}\tabularnewline

\end{longtable}

You are not missing any gene features


\subsection{Missing Product Tags}\label{missing-product-tags}

{{missing_tags_good}} out of {{missing_tags_good + missing_tags_bad}} features have product tags (\texttt{/product="..."}).

The following features are missing product tags
\begin{longtable}{ll}
\hline
Feature & Location\\
\hline
\endhead

{{ row.id | texify }} & \texttt{{'{'}}{{row.location}}{{'}'}}\tabularnewline

\end{longtable}


\subsection{Missing Locus Tags}\label{missing-locus-tags}

{{gene_model_correction_good}} out of {{gene_model_correction_good + gene_model_correction_bad}} features have valid locus tags (\texttt{/locus\_tag="..."}).

The following features have issues with their locus tags
\begin{longtable}{lllll}
\hline
ID & Location & Gene Locus Tag & CDS Locus Tag & Issue \\
\hline
\endhead

{{ row[0].id | texify }} & \texttt{{'{'}}{{row[1].location}}{{'}'}} & {{ row[0].qualifiers['locus_tag'][0] | texify }} & {{ row[1].qualifiers['locus_tag'][0] | texify }} & {{ row[2] | texify }}\tabularnewline

\end{longtable}



\section{Suggested Changes}\label{suggested-changes}

These changes are not required, but are strongly encouraged in order to
provide a uniform genome annotation within the phage community.

\subsection{Start Codons}\label{start-codons}
Nearly all phage genes use ATG, GTG or TTG as start codons. The start codon distribution is as
follows:


\begin{longtable}{lll}
\hline
Start Codon & Count\\
\hline
\endhead

{{ codon_key }} & {{ weird_starts_overall[codon_key] }} \\

\end{longtable}


There are {{weird_starts_bad }} unusual start codons in the genome, these
should be carefully justified. If there is evidence for these starts, the
GenomeA text should note this.

\begin{longtable}{lll}
\hline
Feature ID & Location & Start Codon\\
\hline
\endhead

{{ row.id | texify }} & \texttt{{'{'}}{{row.location}}{{'}'}} & {{row.__start}} \\

\end{longtable}



\subsection{Unannotated RBSs}\label{unannotated-rbss}

The following CDSs either do not have a detectable ribosome binding site (RBS;
Shine-Dalgarno sequence), in which case there is a strong possibility that
this is not the correct start, or there is one but it is not annotated.
Annotating the RBS as part of the gene feature is the best practice.

\begin{longtable}{lllll}
\hline
ID & Location & Error & Upstream (-{{upstream_max}} .. -{{upstream_min}})\\
\hline
\endhead


{{ row.id | texify }} & \texttt{{'{'}}{{row.location}}{{'}'}} & {{row.__message | texify}} & \texttt{{'{'}}{{row.__upstream}}{{'}'}} \\




{{ row.id | texify }} & \texttt{{'{'}}{{row.location}}{{'}'}} & {{row.__message | texify}} & \texttt{{'{'}}{{row.__upstream}}{{'}'}} \\


\end{longtable}

\section{Notes}\label{notes}

These areas may be indicative of a problem, or may simply be
informational. You should examine the areas mentioned in detail.

\subsection{Unusual Gaps}\label{excessive-gaps}

\begin{longtable}{llll}
\hline
Region & Size & Surroundings & Messages\\
\hline
\endhead

\texttt{{'{'}}{{row[0]}}..{{row[1]}}{{'}'}} & {{row[1] - row[0]}} & {{row[2] | nice_strand_tex}} {{row[3] | nice_strand_tex}} &  {{row[4]}} ORFs found in this region \\


\end{longtable}

\subsection{Unusual Overlaps}\label{excessive-overlaps}

\begin{longtable}{llllll}
\hline
\multicolumn{2}{l}{Feature A} & \multicolumn{2}{l}{Feature B} & & \\
ID & Location & ID & Location & Region & Length\\
\hline
\endhead

{{row[0].id | texify}} & \texttt{{'{'}}{{row[0].location}}{{'}'}} & {{row[1].id | texify}} & \texttt{{'{'}}{{row[1].location}}{{'}'}} & {{row[2]}}..{{row[3]}} & {{row[3] - row[2]}} \\

\end{longtable}

\subsection{Coding Density}\label{coding-density}

You have a coding density of {{ coding_density_real }}\% which scores 
{{ coding_density }} / 100 on our scale. Most genomes should be in the 90\% to 100\%
coding density range








\end{document}
